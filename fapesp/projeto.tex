\documentclass[12pt]{extarticle}
\usepackage{latexsym,amsfonts,amsmath,amssymb}
\usepackage[round]{natbib}
\usepackage{setspace}
\usepackage[table]{xcolor}
\usepackage[brazil]{babel}
\usepackage[utf8]{inputenc}
\usepackage[hyphens]{url}
\urlstyle{same}
\usepackage[colorlinks=true,
            citecolor=black,
            urlcolor=blue,
            linkcolor=black,
            breaklinks]{hyperref} 

\usepackage[T1]{fontenc}
\usepackage{libertine}
\renewcommand*\oldstylenums[1]{{\fontfamily{fxlj}\selectfont #1}}

\newcommand{\HRule}{\rule{3.0cm}{0.1mm}}
\newcommand{\fimdesection}{\centerline{\color{green}\rule{3.0cm}{0.1mm} \hspace{0.4cm} $\diamond$ \hspace{0.4cm} \rule{3.0cm}{0.1mm}}}

\pagestyle{myheadings}
\markright{{\scriptsize {\sf \color{blue} Projeto de pesquisa --- Renato M.
Coutinho}\\ \hrulefill }}

\begin{document}

\thispagestyle{empty}

\begin{center}
    \bf \Large \color{blue} PROJETO DE PESQUISA\\
    {\it \small  solicitação de bolsa de Pós--Doutorado}
\end{center}
\vskip 1.0cm
\hfill {\it \underline{submetido à FAPESP}}
\vskip 2.5cm

\setlength{\parindent}{0pt}
\doublespacing
{\sf \fontsize{13pt}{1.5em}\selectfont
Candidato: Renato Mendes Coutinho\\
Supervisor: Paulo Inácio de Knegt López de Prado\\

{\color{green}\hrule}
\vskip 1cm

Instituição: Departamento de Ecologia, Instituto de Biociências, Universidade de São Paulo\\
Nível: Pós--Doutorado\\
Início: 1º de março de 2015\\

{\color{green}\hrule}
\vskip 1cm

Projeto: Dinâmica de metacomunidades em ambientes heterogêneos\\
Palavras--chave: dispersão, metacomunidade, heterogeneidade espacial, dinâmica
de populações, sistemas dinâmicos, modelos matemáticos
}

\newpage

\setlength{\parindent}{20pt}
\thispagestyle{empty}
\begin{center}
    \bf \Large \color{blue} RESUMO
\end{center}
\vskip 3.0cm
{\it

    Para compreender os padrões de diversidade 
    é preciso considerar não só dinâmicas locais, mas como elas 
    se conectam em escala regional.
    Por essa razão, a abordagem de metacomunidades (isto é, comunidades
    ecológicas ligadas por dispersão) é hoje um dos campos mais ativos da ecologia.
    Os ecólogos reconhecem que não há ainda  uma teoria geral de metacomunidades, em parte
    pela falta de modelos matemáticos básicos que resolvam algumas questões fundamentais.
    
    O projeto proposto pretende contribuir para suprir uma das lacunas, 
    que é o desenvolvimento de modelos matemáticos de metacomunidades 
    em espaço explícito heterogêneo.
    Usaremos a teoria e instrumental de análise de sistemas dinâmicos,
    bem como métodos e resultados provenientes da teoria de
    metapopulações e do estudo de módulos de interação da ecologia de
    comunidades. Os objetivos específicos são o desenvolvimento e análise
    de modelos matemáticos que 
    (i) generalizem a capacidade de suporte metapopulacional com taxas heterogêneas para metacomunidades,
    (ii) expressem o efeito de heterogeneidade ambiental sobre interações interespecíficas 
    (com destaque para predação intraguilda), 
    (iii) e mostrem o efeito de atributos da paisagem e das espécies sobre a dinâmica
    de metacomunidades.

%    que ampliem a compreensão dos efeitos
%    recíprocos entre processos locais, como competição e predação, e
%    mecanismos relacionados à diversidade espacial, tais como 
    
%Modelos espacialmente estruturados de metacomunidades: generalizando o
%conceito de capacidade metapopulacional
%
%Estados alternativos e heterogeneidade espacial promovem coexistência de
%metacomunidades: o caso da predação intraguilda
%
%Qual o efeito de características de paisagem sobre a dinâmica de
%metacomunidades?
}

\newpage

\setlength{\parindent}{20pt}
\thispagestyle{empty}
\begin{center}
    \bf \Large \color{blue} ABSTRACT
\end{center}
\vskip 3.0cm
{\it

    In order to understand biodiversity patterns, we need to take into account
    not only local dynamics, but how they connect over a regional scale.
    Because of that, the metacommunity approach (that is, ecological
    communities linked by dispersal) is one of the most active fields in
    ecology nowadays. Ecologists recognize that there is no general
    metacommunity theory yet, in part due to the lack of basic mathematical
    models that deal with some fundamental issues.

    The proposed project aims to fill one of the gaps, which is the
    development of mathematical models of metacommunities in heterogeneous and
    explicitly described space. We will use the theory and toolbox from
    dynamical systems, as well as methods and results coming from
    metapopulation theory, and from the study of food web modules in
    community ecology. The specific goals are the development and analysis of
    mathematical models that
    (i) generalize the concept of metapopulation capacity with heterogeneous
    rates in metacommunities,
    (ii) express the effect of environmental heterogeneity on inter-specific
    interactions (with emphasis on intraguild predation),
    (iii) and display the effect of characteristics of the landscape and of
    the species on the metacommunity dynamics.

%    This project studies the dynamics of metacommunities, i.e.
%    ecological communities connected by dispersal, in situations in which
%    spatial heterogeneity plays an important role in both local and regional
%    dynamics.
%
%    The proposed project aims to explore and extend the theoretical framework
%    of such systems by integrating results from the study of ecological modules,
%    as well as from landscape ecology, to the dynamics of metacommunities. In
%    order to achieve that, we shall develop and analyze new mathematical
%    models, using tools from the study of dynamical systems such as
%    differential equations.

}

\newpage
\setcounter{page}{1}
\onehalfspacing

\section{Introdução}

Metacomunidades são conjuntos de comunidades distribuídas no espaço e
conectadas por dispersão \citep{hanski1991, holyoak2005}. Assim como no
contexto de metapopulações, a dinâmica de comunidades ecológicas locais é
acoplada à dinâmica espacial.  Comunidades, por sua vez, são conjuntos de
espécies que habitam um mesmo lugar e potencialmente interagem entre si. De um
ponto de vista prático, pode ser difícil definir as fronteiras de uma
comunidade, uma vez que diferentes espécies atuam em diferentes escalas de
espaço e de tempo; assim, muitas vezes define-se a comunidade com base em
fronteiras naturais, como lagos, ou no tipo de paisagem (tipicamente a
vegetação predominante).

Embora o termo ``metacomunidade'' seja relativamente recente, ganhando grande
aceitação nas últimas duas décadas, na década de 60 Robert MacArthur e Edward Wilson
já propunham a
teoria de biogeografia de ilhas, que demonstra como o número de espécies em uma comunidade pode estabilizar-se
pelo balanço entre colonização e extinção local {macarthur1967}. 
É da mesma época a teoria clássica de metapopulações \citep{levins1969,levins1971}, 
que mostra como a proporção de manchas ocupadas por uma espécie pode também estabilizar-se pelo 
balanço entre colonização e extinção. Essas teorias foram construídas
independentemente dos estudos de \citet{skellam1951}, que as precederam em
mais de uma década, sobre a dispersão de populações por meio de equações de
reação--difusão.
%% Aqui nestas eção de fundadores vale mencionar o Skellam, que é a aboradgem mais antiga e que usa os modelos de reação-difusão. isso também prepara terreno para as próximas seções, que irão falat de equações RD
%% 
A concepção de que padrões de riqueza e ocupação em sistemas de manchas pode ser influenciada, ou até
determinada, pelo contexto regional por meio do trânsito de indivíduos 
é uma das ideias centrais da ecologia de
comunidades moderna, que atualmente é desenvolvida pelo conceito de
metacomunidade.

Na última década, o estudo tanto teórico quanto empírico de metacomunidades
\citep{logue2011} avançou com o desenvolvimento de um
quadro conceitual unificado.  \citet{leibold2004} enquadraram as diversas
abordagens de estudos de metacomunidades em quatro perspectivas, que não são
mutuamente exclusivas. A primeira delas é a de dinâmica de manchas
(\emph{patches}), conceitualmente muito próxima à teoria clássica de
metapopulações de Levins \citep{levins1969,levins1971}. 
Na perspectiva chamada de %ordenamento de espécies 
\emph{species sorting}, os tipos de recursos determinam a composição local da comunidade,
de modo que diferenças de nicho são mais importantes que a dinâmica espacial
e esta serve, no caso extremo, apenas para manter a composição da comunidade em
cada mancha em sincronia com suas características quando há alterações ambientais. 
%% Seguro que é isto? Minha leitura de species sorting é que as condições seletivas em cada mancha são diferentes, 
%% portanto processos ligados a nicho operam nesta escala, combinando-se com os processos de 
%% migração.
%%%% Você está certo, mas o que eu disse é essencialmente isso: quando você
%%%% tem predominância de efeitos locais, a migração é importante apenas pra
%%%% colonizar as manchas vazias com a espécie mais adaptada àquela mancha.
%%%% Claro que o que eu escrevo aqui é a versão extrema de species sorting,
%%%% é claro que você pode introduzir esses efeitos de maneira que eles tenham
%%%% relativamente pouca importância. Alterei um pouco a frase, vê se fica
%%%% melhor.
%% Ficou ótimo assim, bem geral!
Já na perspectiva de efeitos de massa
(\emph{mass effects}), efeitos de densidade sobre as taxas de colonização, como, por exemplo, o efeito
resgate (\emph{rescue effect}), são fundamentais. As relações entre imigração, emigração e densidades nas manchas
são importantes para descrever a dinâmica da metacomunidade, como no caso de
fontes e sorvedouros (\emph{source-sink dynamics}) \citep{mouquet2002,amarasekare2001}.
%% Colocar 1-2 referências sobre mass effectc e source-sink.
Na perspectiva neutra a riqueza se estabiliza e a composição deriva 
pelo  balanço entre extinções estocásticas e colonizações no nível local 
e especiação no nível regional, em um cenário em que as espécies têm as mesmas
taxas demográficas \citep{hubbell2001}.

Só é possível que o critério de persistência de comunidades no nível da
metacomunidade se distinga do critério de persistência local (com espaço
homogêneo) quando há assincronia temporal entre as comunidades locais, o que
requer que a dispersão das populações seja fraca o suficiente
\citep{chesson1981,amarasekare2003}.
%%citar 1-2 referências. Sugiro alguma das revisões do Leibold e Mouquet & Loreau
%%%% esse ponto não fica claro nas revisões, mas sim nos 2 que coloquei. 
%% ok!
Por outro lado, eventos
de dispersão muito raros não permitiriam que efeitos de massa ocorressem, e
tornaria muito lento o processo de \emph{species sorting}. Dessa maneira,
vemos que as diferentes perspectivas estão, em geral, associadas a certas
escalas de tempo dos processos de dispersão e da dinâmica de metacomunidades.

Apesar do enorme esforço de pesquisa da última década, ainda há muitas
questões inexploradas na teoria de metacomunidades. Este plano de trabalho
aborda o caso de espaço explícito heterogêneo, em que há grande variabilidade entre as
manchas em características tais como tamanho, produtividade e conectividade. O
quadro teórico mais amplo para a compreensão da dinâmica e estabilidade desses
sistemas foi formulado por Peter Chesson, e consiste na aplicação do conceito
de transição de escalas \citep{chesson1981, chesson1998, chesson2005}, que
expõe de maneira clara o papel da covariância entre densidade populacional e
não--linearidade da resposta (por exemplo, em termos do \emph{fitness}) à
variação espacial. Essa teoria é de grande abrangência porque é válida para
regimes em que colonização e extinção locais podem ou não ser importantes para
a dinâmica local. Apesar disso, é interessante abordar modelos específicos,
em que os mecanismos de crescimento populacional são explícitos e é possível
uma conexão clara com características da paisagem.

É importante ressaltar que os modelos desenvolvidos na ecologia, como na
ciência em geral, podem servir a diversas finalidades. Eles formalizam e dão
forma concreta a ideias que, de outra maneira, permaneceriam vagas, servindo
assim para facilitar a comunicação e esclarecer o raciocínio; dessa forma,
permitem a exploração de cenários compatíveis com as hipóteses aventadas.
Podem ainda ter capacidade de previsão em cenários específicos, permitindo o
monitoramento e controle de sistemas, o que os torna importantes para
aplicações. Finalmente, podem unificar conceitos e hipóteses aparentemente
díspares, mostrando que, muitas vezes, processos biológicos distintos atuam
pelos mesmos mecanismos. Modelos em ecologia e evolução em geral não podem 
atender a estes três objetivos igualmente bem, havendo uma demanda conflitante
entre generalidade, realismo e precisão \citep{levins1966}. 
Esta proposta concentra-se no desenvolvimento de modelos gerais e com realismo biológico
que contribuam para o avanço da teoria de metacomunidades.


\subsubsection*{Abordagens matemáticas}

Há uma imensa variedade de abordagens matemáticas usadas na modelagem de
metacomunidades. As principais escolhas referem-se à descrição do espaço e das
variáveis de estado. A estrutura espacial pode ser representada
explicitamente, ou como uma rede de manchas conectadas (incluindo aí o caso
mais simples de apenas duas manchas), ou ainda pode permanecer implícita, caso
em que apenas as taxas (ou probabilidades) de ocupação do total de manchas são
modeladas. Já as variáveis de estado podem ser densidades populacionais,
contínuas, ou números de indivíduos em cada mancha, ou ainda probabilidades
de ocupação de cada mancha.
%% No caso de ocupação as variáveis de estado não seriam a proporção de manchas ocupadas? Taxas e probabilidade me parecem parãmetros
%%%% probabilidade (ou taxa) de ocupação do total de manchas é a mesma coisa que proporção de
%%%% manchas ocupadas, não?
%% Vejo duas diferenças:
%% Probabilidades são valores esperados de proporções (se vc aceita a concepção frequentista de probabilidade), e nesse sentido taxas téoricas. Proporções observadas estão sujeitas à variação em relação a este esperado por efeitos de amostragem e número finito de manchas.
%% mas meu ponto principal é que as proporções são as variáveis de estado, ou seja o que se pretende explicar. Para biólogos isso é é a variável cuja a derivada está à esquerda numa DE, ou que tb chamamos de variáveld dependente. Muitos autores da área usam a palavra "taxa" apenas para os parâmetros ou coeficientes, que estão no lado direito da equação.
%%%% ok, vou deixar só probabilidade
Essas escolhas são fortemente ligadas às
escalas subjacentes e prioridades da modelagem. Por exemplo, um
modelo clássico (tipo Levins) para duas espécies competindo é definido como
\citep{slatkin1974}:

\begin{equation}
  \begin{aligned}
    \frac{dp_1}{dt} &= m_1 (p_1+p_3) p_0 - \left[ e_1 + (m_2-\mu_2)(p_2+p_3) \right] p_1 + (e_2+\epsilon_2)p_3\\
    \frac{dp_2}{dt} &= m_2 (p_2+p_3) p_0 - \left[ e_2 + (m_1-\mu_1)(p_1+p_3) \right] p_2 + (e_1+\epsilon_1)p_3\\
    \frac{dp_3}{dt} &= \left[ (m_1-\mu_1)(p_1+p_3) + (m_2-\mu_2)(p_2+p_3)\right] p_2 -
    (e_1+\epsilon_1+e_2+\epsilon_2)p_3~,
  \end{aligned}
\end{equation}
% 
onde os $p_i$ são as proporções de manchas ocupadas, com $p_0 =
1-(p_1+p_2+p_3)$, e portanto o espaço é implícito. A escala de tempo é tal que detalhes da dinâmica local de cada mancha
são irrelevantes; ao mesmo tempo, assume-se que a dispersão não é suficiente para
sincronizar a dinâmica de toda a região; assume-se ainda que o ambiente é
homogêneo -- no sentido de que todas as manchas são equivalentes -- e que há
extinção estocástica recorrente das populações.
%% Não é importante assinalar que o espaço é implícito?
%%%% eu falei antes que era um modelo tipo Levins, que pra mim diz
%%%% imediatamente que o espaço é implícito, mas acrescentei essa informação agora.

Dadas as premissas acima, a formulação desse modelo não permite
qualquer conclusão sobre fenômenos que envolvam efeitos de massa ou \emph{species sorting}. 
\citet{amarasekare2001} analisam um modelo simples de duas
manchas que incorpora esses dois efeitos:

\begin{equation}
    \begin{aligned}
        \frac{dX_i}{dt} &= r_x X_i \left( 1 - \frac{X_i}{K_{x,i}} - \phi_{x,i}
        \frac{Y_i}{K_{x,i}} \right) + d_x (X_j - X_i), \\
        \frac{dY_i}{dt} &= r_y Y_i \left( 1 - \frac{Y_i}{K_{y,i}} - \phi_{y,i}
        \frac{X_i}{K_{y,i}} \right) + d_y (Y_j - Y_i), \\
        & i, j = 1, 2, \qquad i \neq j~,
    \end{aligned}
\end{equation}
%
onde $X_i$ e $Y_i$ são as abundâncias de cada espécie na mancha $i$. Neste
caso, as densidades são contínuas e a dinâmica populacional é acompanhada numa
escala de tempo mais curta (``local''). Além disso, pode-se introduzir
heterogeneidade espacial escolhendo parâmetros distintos para as manchas $1$ e
$2$. Por outro lado, perde-se a dinâmica de colonização e extinção
estocásticas (embora ainda haja extinção determinística) e a estrutura
espacial da região é reduzida a apenas duas manchas.

Apesar de sua elegância, modelos simples de equações diferenciais como os
apresentados não capturam algumas complexidades da dinâmica espacial que podem
ser de grande importância. Uma das formas de abordar problemas espacialmente
complexos é por meio de equações de reação--difusão \citep{skellam1951,murray2002}:

\begin{equation}
  \begin{aligned}
    \frac{\partial N}{\partial t} &= \nabla^2 \left[ D(\vec{x})N \right] + f(N,P)\\
    \frac{\partial P}{\partial t} &= \nabla^2 \left[ D(\vec{x})P \right] + g(N,P)~,
  \end{aligned}
\end{equation}
%
onde $t$ representa o tempo, $\vec{x}$ a posição no espaço, $N$ e $P$ as
densidades de duas populações, e $\nabla$ é o operador espacial responsável
pela redistribuição espacial da população. Em ambientes heterogêneos, é
essencial analisar cuidadosamente as condições de fronteira, por exemplo,
entre mancha e matriz \citep{turchin1998, ovaskainen2003}. Essa abordagem tem
a vantagem de que o operador $\nabla$ permite a descrição mecanicista da componente
espacial do problema.

Naturalmente, outras formas de introdução do espaço podem ser mais vantajosas,
de acordo com a questão explorada. Por exemplo, \citet{law2000} desenvolveram um
modelo espacial, estocástico e baseado em indivíduos, para comunidades de
plantas e, por meio de uma aproximação, conseguiram reduzi-lo a um sistema mais
fácil de tratar. 
%% reduziram a um sistema de equações de reação-difusão?
%%%% Não, a um sistema de equações integro-diferenciais.
Com isso, eles demonstraram que espécies competitivamente inferiores podem
coexistir, e até mesmo excluir, 
%% superar = exluir ou a ter densidades maiores?
espécies com maior habilidade competitiva que
têm dispersão limitada. Isso é possível quando a espécie superior é mais
aglomerada espacialmente, aumentando sua competição intra--específica com
relação à inter--específica, enquanto que a espécie competitivamente inferior
não sofre o mesmo efeito.

%\subsubsection*{Métodos de análise}

%A elaboração de modelos é útil
%
%Cada abordagem matemática, como as que acabamos de ver, envolve uma
%série de hipóteses mas, principalmente, é apropriada para responder a certas
%questões e não a outras, e daí decorre a análise a ser empregada em cada
%situação. Pretende-se realizar uma análise ampla 
%
%Formulação do modelo \emph{per se}: compreensão de hipóteses mal definidas ou
%empregadas incorretamente, unificação de abordagens


\fimdesection

\section{Objetivos específicos}
\label{sec:objetivos}

\subsubsection*{\em Modelos espacialmente estruturados de metacomunidades:
generalizando o conceito de capacidade metapopulacional}

Embora o conceito de metacomunidades seja derivado do de metapopulações,
muitos dos métodos e ideias deste nunca foram adequadamente traduzidos e
aplicados ao estudo de metacomunidades. Uma das grandes conquistas da teoria
moderna de metapopulações foi a formulação de modelos que são de grande
generalidade, mas também admitem parametrização a partir de dados específicos
de cada paisagem \citep{hanski2000}, o que tem grande valor em aplicações,
especialmente em biologia de conservação.

Baseando-nos em formulações gerais da teoria de metacomunidades
\citep{pillai2010}, visamos estender o modelo desenvolvido por
\citet{ovaskainen2001} para metapopulações, que consiste em introduzir taxas
de colonização e extinção dependentes da mancha. Isto é conveniente para
estudar paisagens específicas, em que as aproximações do modelo de Levins, de
infinitas manchas iguais, não são satisfatórias. Assim, buscamos ampliar o
quadro teórico de metacomunidades a fim de obter maior capacidade preditiva e
permitir contato mais próximo com medidas observacionais e experimentais.

\subsubsection*{\em Estados alternativos e heterogeneidade espacial promovem
coexistência de metacomunidades: o caso da predação intraguilda}

Heterogeneidade espacial é um dos principais fatores capazes de promover a
coexistência de espécies \citep{amarasekare2003}, sobretudo devido ao
mecanismo de \emph{species sorting}, em que as hierarquias competitivas variam entre manchas, 
devido a diferenças nas condições e recursos das manchas. 
Propomos explorar essa ideia para compreender melhor
como a variação entre manchas de \emph{habitat} promove a estabilidade de
redes tróficas simples.

Em particular, pretendemos estudar o caso da predação intraguilda, um módulo
de grande importância de um ponto de vista fundamental, já que faz parte de
qualquer rede trófica, e pode provocar grande variação no comprimento e na
produtividade de cadeias tróficas, além de ter um papel importante na
manutenção da diversidade de espécies em largas escalas de espaço
\citep{mccann2011}. Num sistema com predação intraguilda, o resultado da
dinâmica (ou seja, a persistência ou não das espécies) depende da
produtividade de recursos: sob baixa produtividade, apenas a presa intraguilda
persiste, enquanto que em altas produtividades o predador intraguilda é capaz
de excluí-la, e há coexistência estável apenas em produtividades
intermediárias \citep{holt1997}. Essa conclusão é problemática na ecologia de
populações tradicional, já que a predação intraguilda é ubíqua, mas, de acordo
com a teoria, ela persistiria apenas para uma faixa de parâmetros muito
restrita.

Aqui, tomamos como objetivo desenvolver modelos que incorporem explicitamente
paisagens contendo manchas de diferentes produtividades, e explorar em que
condições se observa coexistência estável.

\subsubsection*{\em Qual o efeito de características de paisagem sobre a
dinâmica de metacomunidades?}

Finalmente, buscamos explorar como fatores físicos e comportamentais, como
tamanho de manchas e a reação de organismos à borda da mancha com a  matriz, %borda é um termo mais caro aos ecólogos de paisagem
pode ter consequências para a distribuição espacial dessas espécies em
diferentes escalas espaciais. A princípio, isto deve ser explorado em um
contexto mais simples, de metapopulações \citep{ovaskainen2004} (ou
``metacomunidades'' sem interações entre espécies), que deve fornecer uma base
sólida para estender essa abordagem para sistemas com interações entre as
espécies. 


%\subsubsection*{\em Predação intraguilda em manchas de produtividade heterogênea}
%
%Em primeiro lugar, escolhemos um módulo de interação -- a predação intraguilda
%-- que é de grande importância de um ponto de vista fundamental, já que faz
%parte de qualquer rede trófica, e pode provocar grande variação no comprimento
%e na produtividade de cadeias tróficas, além de ter um possível papel
%importante na manutenção da diversidade de espécies em largas escalas de
%espaço \citep{mccann2011}.
%
%A predação intraguilda é de particular interesse para este projeto porque a
%teoria tradicional prevê que o resultado da sua dinâmica depende da
%produtividade do sistema: sob baixa produtividade, apenas a presa intraguilda
%persiste, enquanto que em altas produtividades o predador intraguilda é capaz
%de excluí-la, e há coexistência estável apenas em produtividades
%intermediárias \citep{holt1997}. Essa conclusão é problemática na ecologia de
%populações tradicional, já que a predação intraguilda é ubíqua, mas, de acordo
%com a teoria, ela persistiria apenas para uma faixa de parâmetros muito
%restrita.
%
%Aqui, tomamos como objetivo desenvolver métodos para incorporar explicitamente
%paisagens contendo manchas de diferentes produtividades, e explorar em que
%condições se observa coexistência estável.
%
%\subsubsection*{\em Características de paisagem em metacomunidades}
%
%Este segundo tópico busca explorar como fatores físicos e comportamentais,
%como tamanho de manchas e a reação de organismos à fronteira entre mancha e
%matriz, pode ter consequências para a distribuição espacial dessas espécies em
%diferentes escalas espaciais. A princípio, isto deve ser explorado em um
%contexto mais simples, de metapopulações (ou ``metacomunidades'' sem
%interações entre espécies), que deve fornecer uma base sólida para estender
%essa abordagem para sistemas com interações entre as espécies. 

\fimdesection

\section{Metodologia} % Procedimentos?
%% Esta é parte que demanda mais atenção, veja comentários abaixo.
%% O que te sugiro trabalhar é tranquilizar o parecerista de que há um método e qual é ele.
%% Um roteiro:
%% 1 - Definir a abordagem analítica matemática e distinguí-la da de simulação (que é a que vem à cabeça de biólogos quando se fala de métodos computacionais). Aqui ajuda citar os principais métodos computacionais disponíveis for dummies (integração numérica, etc).
%% 2 - Explicar numa sequencia de passos for dummies o que é desenvolver um modelo matemático com auxílio destas ferramnetas. Aqui cabem tanto as etapas básicas como em cada uma alguma tecnicalidade de como é feito (e.g., crição de classes em Python e seus métodos para realizar os cálculos, etc).
%% 3 - Indicar nestes passos os critérios para avaliar se os objetivos proposto para modelo foi alcançado.
%% Sei que pode parecer estranho tanto detalhe, mas há uma cultura na bio de que métodologia tem que ter um protocolo mais definido.

%% te sugiro avaliar se é preciso distinguir abordagem analítica do que vc chama de computacional, que para biólogos parece a mesma coisa. Por outro lado, abordagem computacional para biólogos são simulações computacionais da dinâmica, como no caso de ABMs, autômatos celulares, etc. Esta é a distinção crucial para nós em relação ao que chamamos de métodos matemáticos, balaio no qual jogamos tanto soluções analíticas senso estrito como as que usam aproximações numéricas computacionais.
%% Nós colocamos no mesmo balaio porque não nos parece haver muita diferença qualitativa entre uma análise qualitativa (estados estacionários, equilíbrio, etc) analítica ou numérica/computacional.
%% A distinção aqui vai servir apenas para pontuar que vc vai usar a abordagem matemática e não a de simulações.
%% Em seguida, e mais importante de tudo: um resumo das etapas de desenvolvimento do modelo, na forma de bullets, para que não digam que falta uma receita ao projeto. Biólogos tem horro a metodologias muito abertas, por estranho que te pareça.

Neste projeto, estamos interessados no comportamento de sistemas teóricos no nível
populacional, isto é, buscamos condensar a expressão de processos que ocorrem
na escala de indivíduos em efeitos que ocorrem no nível de populações
\citep{turchin2003} ou no nível acima, de metapopulações e metacomunidades.
Isso nos leva a considerar modelos dinâmicos cujas variáveis são propriedades das populações,
representadas mais naturalmente por suas taxas de variação em equações diferenciais ou a diferenças.
Com esses modelos matemáticos, as previsões são obtidas usando técnicas analíticas e aproximações
numéricas provenientes da teoria de sistemas dinâmicos, em contraste com
modelos computacionais, em que o comportamento de cada indivíduo é modelado, e
a partir disso simula-se o sistema.

A elaboração de cada modelo é um processo iterativo, que parte em geral de
modelos simples com propriedades bem conhecidas. Diversos procedimentos de análise são então usados
para desenvolver mudanças no modelo para alcançar um certo objetivo, e também para
avaliar as consequências destas mudanças no comportamento do modelo.
Buscamos a construção de conteúdos teóricos novos, mas que pode ser
caracterizada de maneira geral da seguinte forma:

\renewcommand{\thesubsubsection}{\arabic{subsubsection}} % sem número de seção

\subsubsection{Formulação}

Levando em conta os objetivos, escolhemos o nível de descrição adequado, isto
é, como são descritas as variáveis independentes (tempo e espaço) e o estado
do sistema (populações). Por exemplo, se estamos interessados em fazer
previsões sobre paisagens específicas, incluindo o papel de tamanho e
conectividade de cada mancha, é necessário que o espaço figure explicitamente
no modelo.

Precisamos ainda especificar quais processos serão levados em conta.  Neste
projeto, especificamente, consideramos processos como migração, colonização e
extinção, bem como interações entre espécies -- competição e predação. Em
modelos de equações diferenciais, a cada processo corresponde um termo que
representa sua contribuição instantânea ao estado do sistema. É importante que
as escalas de tempo e espaço desses termos sejam compatíveis com a descrição
determinada previamente. Quando modelamos extinção estocástica, por exemplo,
numa descrição em que o espaço está implícito, podemos simplesmente
acrescentar uma taxa constante por mancha -- ``determinística'' --, como no
modelo de metapopulações de Levins \citep{levins1969}; já numa situação em que
acompanhamos as densidades populacionais em cada mancha e em escala de tempo
curta, esse mesmo processo deve ser modelado usando um termo estocástico.
% como expressar melhor ``mais simples e mais complicada''?
% mover para o início da subseção?
% Me parece muito bom assim, e neste lugar do texto
Esta etapa é, ao mesmo tempo, a mais simples e a mais complicada pois, embora
a expressão de um sistema de equações seja sucinta, a intuição de quais
opções serão mais produtivas e parcimoniosas exige experiência prévia com uma
gama de modelos e abordagens.

\subsubsection{Análise}

Dado o modelo matemático que desejamos estudar, empregamos métodos analíticos
e aproximações numéricas apropriadas para sua análise, isto é, para determinar
o comportamento dinâmico e o estado de longo prazo do sistema em função de
detalhes específicos a cada situação -- os parâmetros do modelo.

O método mais tradicional de análise consiste na chamada análise qualitativa
de sistemas de equações diferenciais, em que se calculam as soluções
estacionárias do problema e sua estabilidade, e a partir disso pode-se
compreender a dinâmica do sistema como um todo \citep{murray2002,gurney1998}.
Em problemas espacialmente explícitos, tais métodos são de difícil
implementação, mas existem métodos de homogeneização que, por vezes, oferecem
boas aproximações \citep{cobbold2014}. Em muitos casos, alguns cálculos
envolvidos, tais como encontrar soluções para sistemas de equações algébricas
ou os autovalores de matrizes, existem mas são intratáveis na prática, ou tomariam
muito tempo.
Hoje dispomos de rotinas computacionais de matemática simbólica que realizam este
trabalho de maneira mais rápida e confiável, como a poderosa
biblioteca SymPy \citep{sympy}, que usaremos.

Muitas vezes, lançamos mão de métodos de aproximação numérica das soluções,
especialmente quando precisamos determinar a variação com o tempo das soluções
(em contraste com o caso em que buscamos apenas saber o estado final).  Usamos
ferramentas do cálculo numérico \citep{iserles2009}, em geral já implementadas
em ferramentas computacionais como o SciPy \citep{scipy}, em Python, que provê
rotinas muito robustas para a integração numérica de sistemas de equações
diferenciais. Em sistemas de equações sem espaço, escrevemos funções que
correspondem às equações do sistema estudado, fornecemos os parâmetros e o
estado inicial, e invocamos uma função da biblioteca que efetua a integração
numérica e retorna a solução em função do tempo. Já em sistemas em que a
representação do espaço é contínua, o procedimento é mais complexo, e devemos
aproximar a distribuição espacial por um número finito, porém grande, de
segmentos no espaço, e determinar equações aproximadas para o estado de cada
um deles \citep{iserles2009}, que serão então integradas usando o mesmo método
de antes.

\subsubsection{Interpretação e síntese}

% difícil..
% Sim, a mais difícil, mas acho que vale refletir mais sobre isso.
%% Em primeiro lugar, sobre a relação entre o resultado matemático e sua interpretação
%% Não diria que com o resultado da análise ele é possível compreender o mecanismo real, mas sim
%% quais as consequência dos mecanismos propostos sobre
%% a dinâmica do sistema, tal como descrita pelo modelo.
%% Em segundo há a questão da generalidade de cada conclusão destas. 
%% Aqui caberia mencionar que a avaliação de generalidade é feita identificando-se em espaços de fase 
%% regiões que caracterizam um certa propriedade de interesse biológico (pe.ex
%% um regime, pontos estacionários não triviais, estados estáveis, limiares).
%% Feito isso seria possível enunciar a regras de relevância biológica que regem
%% a dinâmica, sob um dado modelo.
%% Terceiro, vai ajudar muito o assessor se aqui você der alguns exemplos de como a interpretação
%% pode ser feita nos estudos de caso propostos. Por exemplo, vc pode lembrar o que vai se buscar nos
%% resultados de cada modelo, e/ou indicar os espaços de fase relevantes.


Esta última etapa consiste em extrair das análises e hipóteses do modelo o
máximo de informação de interesse biológico, com vistas a estender nossa
compreensão dos processos e mecanismos envolvidos, assim como nossa capacidade
de prever cenários e delinear novos testes da teoria. Para tanto, é necessário
clareza na relação entre os processos introduzidos e os parâmetros do modelo.

%As questões de interesse ecológico só
%podem ser respondidas por modelos quando é possível estabelecer uma relação
%clara entre os processos introduzidos no modelo e seus parâmetros.
%É a partir dessa compreensão que somos capazes de julgar se as previsões do
%modelo correspondem ao que era esperado intuitivamente, ou ao que já era
%conhecido empiricamente, e assim podemos idealizar novos experimentos e testes
%da teoria.

Um aspecto básico a ser considerado é delimitar os diferentes regimes
previstos, que correspondem, na linguagem de sistemas dinâmicos, em
identificar a quais valores de parâmetros correspondem a estabilidade de
pontos de equilíbrio do sistema (e.g. a estabilidade da solução trivial, com
população zero, equivale à extinção) ou características da solução (e.g.
ciclos populacionais). As transições entre regimes ocorrem quando limiares são
ultrapassados, e muitos êxitos da teoria ecológica residem em determinar quais
características influenciam esses limiares (e.g. \citet{hanski2000,
fussmann2000}).

Essa análise do chamado espaço de parâmetros também é importante para avaliar
a generalidade de um certo resultado: a expectativa é que regimes que existem
apenas sob condições ou parâmetros muito restritos sejam raros na natureza.
Por exemplo, um dos problemas da teoria tradicional de predação intraguilda,
desenvolvida por \citet{holt1997}, é o fato de a coexistência entre presa e
predador intraguilda ocorrer apenas para uma faixa de produtividade ambiental
muito pequena, o que contradiz ampla evidência experimental \citep{arim2004}
e induziu muitos trabalhos teóricos que buscam encontrar os processos que
promovem coexistência porém são negligenciados pelos modelos mais simples (e.g.
\citet{holt2007, amarasekare2008}).

Por vezes, o modelo esclarece o mecanismo que leva as hipóteses assumidas no
início às conclusões obtidas. Em seu artigo seminal, \citet{skellam1951}
mostra como uma região de \emph{habitat} cercada por uma matriz completamente
inóspita só é capaz de suportar uma população que cresce e se difunde se for
maior que uma área mínima. Sua conclusão vai além do modelo específico
empregado, e mostra que isso se deve aos processos de dispersão de indivíduos
para a matriz (relacionado ao perímetro da região e à difusividade da
espécie), e de crescimento populacional (relacionado à área e à taxa de
crescimento), e portanto o limiar de persistência da população é determinado
por relações entre perímetro e área da mancha e entre difusividade e taxa de
crescimento intrínseco da população.

\fimdesection

\section{Cronograma de trabalho}
% Se os passos gerais de elaboração de cada modelo estiverem claros na seção acima estas considerações aqui são desnecessárias.
% O importante é pensar numa descrição


%É sempre difícil apresentar um cronograma de trabalho preciso em projetos de natureza teórica. 
% O primeiro passo será aprofundar-se na literatura pertinente
% e suas principais classes de modelos. A seguir, iremos
% desenvolver e refinar modelos explorando os problemas propostos na seção
% \ref{sec:objetivos}. Até o fim do ano, esperamos obter resultados suficientes
% para publicação.

% A elaboração de cada modelo é um processo criativo, que envolve idas e vindas,
% já que em cada passo é necessário analisar as hipóteses e suas consequências,
% usando os métodos analíticos e computacionais descritos, alterando formas de
% descrição, hipóteses e análises, até alcançar modelos que sejam informativos,
% porém suficientemente simples para que sua interpretação seja clara. 

% Naturalmente, o candidato deve dar continuidade a colaborações nacionais e
% internacionais que têm se mostrado produtivas.

\begin{table}[h!] 
\footnotesize
\noindent
\begin{tabular}{p{5.4cm}|p{0.2cm}p{0.2cm}p{0.2cm}p{0.2cm}p{0.2cm}p{0.2cm}p{0.2cm}p{0.2cm}p{0.2cm}p{0.2cm}p{0.2cm}p{0.2cm}}
    {\normalsize Atividades\textbackslash\hspace{0.1cm} Bimestre} & 1 & 2 & 3 & 4 & 5 & 6 & 7 & 8
                                                                & 9 & 10 & 11 & 12 \\ \hline 
                                              revisão da literatura& \cellcolor[gray]{0.8} & \cellcolor[gray]{0.8} &  &  &  &  &  &  &  &  &  &   \\
                                    modelo predação intraguilda &
                                          \cellcolor[gray]{0.8} &
                                          \cellcolor[gray]{0.8} &
                                          \cellcolor[gray]{0.8} &
                                          \cellcolor[gray]{0.8} &
                                          \cellcolor[gray]{0.8} & & & & & & & \\
                                    \mbox{modelos espacialmente estruturados} & & & &
                                          \cellcolor[gray]{0.8} &
                                          \cellcolor[gray]{0.8} &
                                          \cellcolor[gray]{0.8} &
                                          \cellcolor[gray]{0.8} &
                                          \cellcolor[gray]{0.8} & & & & \\
                                    \mbox{modelos de paisagens para metacom.} & & & & & & &
                                          \cellcolor[gray]{0.8} &
                                          \cellcolor[gray]{0.8} &
                                          \cellcolor[gray]{0.8} &
                                          \cellcolor[gray]{0.8} &
                                          \cellcolor[gray]{0.8} & \\
                                     \mbox{exploração conexão entre modelos} & & &
                                          \cellcolor[gray]{0.8} &
                                          \cellcolor[gray]{0.8} &
                                          \cellcolor[gray]{0.8} &
                                          \cellcolor[gray]{0.8} &
                                          \cellcolor[gray]{0.8} &
                                          \cellcolor[gray]{0.8} &
                                          \cellcolor[gray]{0.8} &
                                          \cellcolor[gray]{0.8} &
                                          \cellcolor[gray]{0.8} & \\ 
                                    escrita dos resultados & & & & & 
                                          \cellcolor[gray]{0.8} &
                                          \cellcolor[gray]{0.8} & & & & &
                                          \cellcolor[gray]{0.8} &
    \cellcolor[gray]{0.8} \\
\end{tabular}
\end{table}

\fimdesection

\section{Considerações finais} % justificativa?

O candidato desenvolve desde sua graduação pesquisa em sistemas dinâmicos em
biologia, buscando ativamente colaboração com pesquisadores da ecologia. Entre
os resultados há artigos em periódicos de alto impacto da área
\citep{coutinho2012, laporta2013, assaneo2013, fonseca2013, amarasekare2013,
moretti2013, azevedo2014, amarasekare2014} e participação como professor
assistente nas \emph{Southern Summer Schools on Mathematical Biology}
\citep{SSSMB2012, SSSMB2013, SSSMB2014}, que promovem a interação entre
estudantes das ciências exatas e biológicas. O supervisor da proposta é hoje
um dos colaboradores principais do candidato e busca, igualmente, impulsionar
o diálogo com pesquisadores das ciências exatas para avanço da teoria. A
presente proposta tem o objetivo de estreitar esta colaboração e consolidar a
formação do candidato na área de ecologia teórica, com benefícios de parte a
parte.

Este projeto fará uso de diversas técnicas provenientes da teoria de sistemas
dinâmicos para avançar a compreensão de metacomunidades. Assim, a formação
interdisciplinar do candidato é uma grande oportunidade, já que a familiaridade com
modelos em ecologia permite navegar nessa área, que apresenta uma vasta gama de
abordagens teóricas e de argumentos e hipóteses cuja tradução matemática pode
ser sutil, e ainda está por desenvolver-se \citet{leibold2011}.

\fimdesection

\bibliographystyle{amnatnat-pt}
\bibliography{referencias,minhas_citacoes}

\vskip 3.0cm
{\hfill S\~ao Paulo, \today}

{\hfill  \it Paulo Inácio de Knegt López de Prado}

{\hfill  \it Renato Mendes Coutinho}

\end{document}

