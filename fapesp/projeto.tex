\documentclass[12pt]{extarticle}
\usepackage{latexsym,amsfonts,amsmath,amssymb}
\usepackage[round]{natbib}
\usepackage{setspace}
\usepackage[table]{xcolor}
\usepackage[brazil]{babel}
\usepackage[utf8]{inputenc}
\usepackage[hyphens]{url}
\urlstyle{same}
\usepackage[colorlinks=true,
            citecolor=black,
            urlcolor=blue,
            linkcolor=black,
            breaklinks]{hyperref} 

\usepackage[T1]{fontenc}
\usepackage{libertine}
\renewcommand*\oldstylenums[1]{{\fontfamily{fxlj}\selectfont #1}}

\newcommand{\HRule}{\rule{3.0cm}{0.1mm}}
\newcommand{\fimdesection}{\centerline{\color{green}\rule{3.0cm}{0.1mm} \hspace{0.4cm} $\diamond$ \hspace{0.4cm} \rule{3.0cm}{0.1mm}}}

\pagestyle{myheadings}
\markright{{\scriptsize {\sf \color{blue} Projeto de pesquisa --- Renato M.
Coutinho}\\ \hrulefill }}

\begin{document}

\thispagestyle{empty}

\begin{center}
    \bf \Large \color{blue} PROJETO DE PESQUISA\\
    {\it \small  solicitação de bolsa de Pós--Doutorado}
\end{center}
\vskip 1.0cm
\hfill {\it \underline{submetido à FAPESP}}
\vskip 2.5cm

\setlength{\parindent}{0pt}
\doublespacing
{\sf \fontsize{13pt}{1.5em}\selectfont
Candidato: Renato Mendes Coutinho\\
Supervisor: Paulo Inácio de Knegt López de Prado\\

{\color{green}\hrule}
\vskip 1cm

Instituição: Departamento de Ecologia, Instituto de Biociências, Universidade de São Paulo\\
Nível: Pós--Doutorado\\
Início: 1º de março de 2015\\

{\color{green}\hrule}
\vskip 1cm

Projeto: Dinâmica de metacomunidades em ambientes heterogêneos\\
Palavras--chave: dispersão, metacomunidade, heterogeneidade espacial, dinâmica
de populações, modelos matemáticos
}

\newpage

\setlength{\parindent}{20pt}
\thispagestyle{empty}
\begin{center}
    \bf \Large \color{blue} RESUMO
\end{center}
\vskip 3.0cm
{\it

    Este projeto estuda a dinâmica de metacomunidades, isto é, comunidades
    ecológicas conectadas por dispersão, em contextos em que a heterogeneidade
    espacial exerce papel importante tanto na dinâmica local quanto na
    dinâmica regional.

    O projeto proposto pretende explorar e estender o quadro teórico desses
    sistemas, incorporando métodos e resultados provenientes da teoria de
    metapopulações do estudo de módulos de interação da ecologia de
    comunidades e à dinâmica de metacomunidades.  Para isso, deve-se
    desenvolver e analisar novos modelos matemáticos, usando ferramentas de
    sistemas dinâmicos, como sistemas de equações diferenciais.

%    que ampliem a compreensão dos efeitos
%    recíprocos entre processos locais, como competição e predação, e
%    mecanismos relacionados à diversidade espacial, tais como 
    
%Modelos espacialmente estruturados de metacomunidades: generalizando o
%conceito de capacidade metapopulacional
%
%Estados alternativos e heterogeneidade espacial promovem coexistência de
%metacomunidades: o caso da predação intraguilda
%
%Qual o efeito de características de paisagem sobre a dinâmica de
%metacomunidades?
}

\newpage

\setlength{\parindent}{20pt}
\thispagestyle{empty}
\begin{center}
    \bf \Large \color{blue} ABSTRACT
\end{center}
\vskip 3.0cm
{\it

    [TODO: atualizar tradução]

    This project studies the dynamics of metacommunities, i.e.
    ecological communities connected by dispersal, in situations in which
    spatial heterogeneity plays an important role in both local and regional
    dynamics.

    The proposed project aims to explore and extend the theoretical framework
    of such systems by integrating results from the study of ecological modules,
    as well as from landscape ecology, to the dynamics of metacommunities.
}

\newpage
\setcounter{page}{1}
\onehalfspacing

\section{Introdução}

Metacomunidades são conjuntos de comunidades distribuídas no espaço e
conectadas por dispersão \citep{hanski1991, holyoak2005}. Assim como no
contexto de metapopulações, a dinâmica de comunidades ecológicas locais é
acoplada à dinâmica espacial.  Comunidades, por sua vez, são conjuntos de
espécies que habitam um mesmo lugar e potencialmente interagem entre si. De um
ponto de vista prático, pode ser difícil definir as fronteiras de uma
comunidade, uma vez que diferentes espécies atuam em diferentes escalas de
espaço e de tempo; assim, muitas vezes define-se a comunidade com base em
fronteiras naturais, como lagos, ou no tipo de paisagem (tipicamente a
vegetação predominante).

Embora o termo ``metacomunidade'' seja relativamente recente, ganhando grande
aceitação nas últimas duas décadas, na década de 60 MacArthur já propunha a
teoria de biogeografia de ilhas, que explica padrões de diversidade a partir
do tamanho e da distância da ilha ao continente \citep{macarthur1967}. A
concepção de que a composição de uma comunidade pode ser influenciada, ou até
determinada, pelo contexto regional é uma das ideias centrais da ecologia de
comunidades moderna, e é daí que vem a importância do conceito de
metacomunidade.

Na última década, o estudo tanto teórico quanto empírico de metacomunidades
\citep{logue2011} recebeu grande impulso a partir do desenvolvimento de um
quadro conceitual unificado.  \citet{leibold2004} enquadraram as diversas
abordagens de estudos de metacomunidades em quatro perspectivas, que não são
mutuamente exclusivas. A primeira delas é a de dinâmica de manchas
(\emph{patches}), conceitualmente muito próxima à teoria clássica de
metapopulações de Levins \citep{levins1969,levins1971}, que modela apenas
a proporção de manchas ocupadas, determinada pelos processos de colonização
e extinção. Na perspectiva chamada de ordenamento de espécies (\emph{species
sorting}), os tipos de recursos determinam a composição local da comunidade,
de modo que diferenças de nicho são mais importantes que a dinâmica espacial,
porém esta deve estar presente para que a composição da comunidade acompanhe
possíveis alterações ambientais.  Já na perspectiva de efeitos de massa
(\emph{mass effects}), efeitos de densidade, como, por exemplo, o efeito
resgate (\emph{rescue effect}), são fundamentais. Imigração e emigração têm
forte impacto na densidade local, que em geral é o foco.  Finalmente, na
perspectiva neutra todas as espécies são ecologicamente idênticas, e a
dinâmica é determinada pelos processos de emigração, imigração, extinção e
especiação. A composição das comunidades muda por deriva ecológica
(\emph{ecological drift}).

Só é possível que o critério de persistência de comunidades no nível da
metacomunidade se distingua do critério de persistência local (com espaço
homogêneo) quando há assincronia temporal entre as comunidades locais, o que
requer que a dispersão das populações seja incompleta. Por outro lado, eventos
de dispersão muito raros não permitiriam que efeitos de massa ocorressem, e
tornaria muito lento o processo de ordenamento de espécies. Dessa maneira,
vemos que as diferentes perspectivas estão, em geral, associadas a certas
escalas de tempo dos processos de dispersão e da dinâmica de metacomunidades.

Apesar do enorme esforço de pesquisa da última década, ainda há muitas
questões inexploradas na teoria de metacomunidades. Este plano de trabalho
foca em ambientes heterogêneos, em que há grande variabilidade entre as
manchas em características tais como tamanho, produtividade e conectividade. O
quadro teórico mais amplo para a compreensão da dinâmica e estabilidade desses
sistemas foi formulado por Peter Chesson, e consiste na aplicação do conceito
de transição de escalas \citep{chesson1981, chesson1998, chesson2005}, que
expõe de maneira clara o papel da covariância entre densidade populacional e
não--linearidade da resposta (por exemplo, em termos do \emph{fitness}) à
variação espacial. Essa teoria é de grande abrangência porque é válida para
regimes em que colonização e extinção locais podem ou não ser importantes para
a dinâmica local. Apesar disso, é interessante abordar modelos específicos,
em que os mecanismos de crescimento populacional são explícitos e é possível
uma conexão clara com características da paisagem.

É importante ressaltar que os modelos desenvolvidos na ecologia, como na
ciência em geral, podem servir a diversas finalidades. Eles formalizam e dão
forma concreta a ideias que, de outra maneira, permaneceriam vagas, servindo
assim para facilitar a comunicação e esclarecer o raciocínio; dessa forma,
permitem a exploração de cenários compatíveis com as hipóteses aventadas.
Podem ainda ter capacidade de previsão em cenários específicos, permitindo o
monitoramento e controle de sistemas, o que os torna importantes para
aplicações. Finalmente, podem unificar conceitos e hipóteses aparentemente
díspares, mostrando que, muitas vezes, processos biológicos distintos atuam
pelos mesmos mecanismos.

\fimdesection

\section{Métodos}

\subsubsection*{Abordagens matemáticas}

Há uma imensa variedade de abordagens matemáticas usadas na modelagem de
metacomunidades. As principais escolhas referem-se à descrição do espaço e das
variáveis de estado. A estrutura espacial pode ser representada
explicitamente, ou como uma rede de manchas conectadas (incluindo aí o caso
mais simples de apenas duas manchas), ou ainda pode permanecer implícita, caso
em que apenas as taxas (ou probabilidades) de ocupação do total de manchas são
modeladas. Já as variáveis de estado podem ser densidades populacionais,
contínuas, ou números de indivíduos em cada mancha, ou ainda probabilidades
(ou taxas) de ocupação de manchas. Essas escolhas são fortemente ligadas às
escalas subjacentes e simplificações do sistema sendo tratado. Por exemplo, um
modelo clássico (tipo Levins) para duas espécies competindo é definido como
\citep{slatkin1974}:

\begin{equation}
  \begin{aligned}
    \frac{dp_1}{dt} &= m_1 (p_1+p_3) p_0 - \left[ e_1 + (m_2-\mu_2)(p_2+p_3) \right] p_1 + (e_2+\epsilon_2)p_3\\
    \frac{dp_2}{dt} &= m_2 (p_2+p_3) p_0 - \left[ e_2 + (m_1-\mu_1)(p_1+p_3) \right] p_2 + (e_1+\epsilon_1)p_3\\
    \frac{dp_3}{dt} &= \left[ (m_1-\mu_1)(p_1+p_3) + (m_2-\mu_2)(p_2+p_3)\right] p_2 -
    (e_1+\epsilon_1+e_2+\epsilon_2)p_3~,
  \end{aligned}
\end{equation}
% 
onde os $p_i$ são as proporções de manchas ocupadas, com $p_0 =
1-(p_1+p_2+p_3)$. A escala de tempo é tal que detalhes da dinâmica local de cada mancha
são irrelevantes; ao mesmo tempo, assume-se que a dispersão não é suficiente para
sincronizar a dinâmica de toda a região; assume-se ainda que o ambiente é
homogêneo -- no sentido de que todas as manchas são equivalentes -- e que há
extinção estocástica recorrente das populações.

Dadas as hipóteses consideradas acima, a formulação desse modelo não permite
qualquer conclusão sobre fenômenos que envolvam efeitos de massa ou ordenação
de espécies. \citet{amarasekare2001} analisam um modelo simples de duas
manchas que incorpora esses dois efeitos:

\begin{equation}
    \begin{aligned}
        \frac{dX_i}{dt} &= r_x X_i \left( 1 - \frac{X_i}{K_{x,i}} - \phi_{x,i}
        \frac{Y_i}{K_{x,i}} \right) + d_x (X_j - X_i), \\
        \frac{dY_i}{dt} &= r_y Y_i \left( 1 - \frac{Y_i}{K_{y,i}} - \phi_{y,i}
        \frac{X_i}{K_{y,i}} \right) + d_y (Y_j - Y_i), \\
        & i, j = 1, 2, \qquad i \neq j~,
    \end{aligned}
\end{equation}
onde $X_i$ e $Y_i$ são as abundâncias de cada espécie na mancha $i$. Neste
caso, as densidades são contínuas e a dinâmica populacional é acompanhada numa
escala de tempo mais curta (``local''). Além disso, pode-se introduzir
heterogeneidade espacial escolhendo parâmetros distintos para as manchas $1$ e
$2$. Por outro lado, perde-se a dinâmica de colonização e extinção
estocásticas (embora ainda haja extinção determinística) e a estrutura
espacial da região é reduzida a apenas duas manchas.

Apesar de sua elegância, modelos simples de equações diferenciais como os
apresentados não capturam algumas complexidades da dinâmica espacial que podem
ser de grande importância. Uma das formas de abordar problemas espacialmente
complexos é por meio de equações de reação--difusão \citep{murray2002}:

\begin{equation}
  \begin{aligned}
    \frac{\partial N}{\partial t} &= \nabla^2 \left[ D(\vec{x})N \right] + f(N,P)\\
    \frac{\partial P}{\partial t} &= \nabla^2 \left[ D(\vec{x})P \right] + g(N,P)~,
  \end{aligned}
\end{equation}
%
onde $t$ representa o tempo, $\vec{x}$ a posição no espaço, $N$ e $P$ as
densidades de duas populações, e $\nabla$ é o operador espacial responsável
pela redistribuição espacial da população. Em ambientes heterogêneos, é
essencial analisar cuidadosamente as condições de fronteira, por exemplo,
entre mancha e matriz \citep{turchin1998, ovaskainen2003}. Essa abordagem tem
a vantagem de ser estritamente mecanicista no que diz respeito à componente
espacial do problema.

Naturalmente, outras formas de introdução do espaço podem ser mais vantajosas,
de acordo com a questão explorada. Por exemplo, \citet{law2000} desenvolvem um
modelo espacial, estocástico e baseado em indivíduos, para comunidades de
plantas e, por meio de uma aproximação, conseguem reduzi-lo a um sistema mais
fácil de tratar. Com isso, eles conseguem analisar um fenômeno importante do
sistema estudado: espécies competitivamente inferiores podem ser capazes de
coexistir, e até mesmo superar, espécies com maior habilidade competitiva que
têm dispersão limitada. Isso é possível porque a espécie superior é mais
aglomerada espacialmente, aumentando sua competição intra--específica com
relação à inter--específica, enquanto que a espécie competitivamente inferior
não sofre o mesmo efeito.

\subsubsection*{Métodos de análise}

%A elaboração de modelos é útil
%
%Cada abordagem matemática, como as que acabamos de ver, envolve uma
%série de hipóteses mas, principalmente, é apropriada para responder a certas
%questões e não a outras, e daí decorre a análise a ser empregada em cada
%situação. Pretende-se realizar uma análise ampla 
%
%Formulação do modelo \emph{per se}: compreensão de hipóteses mal definidas ou
%empregadas incorretamente, unificação de abordagens


Os métodos tradicionais para a análise de modelos matemáticos como os que
acabamos de ver se dividem em duas categorias: técnicas
analíticas e técnicas computacionais. As primeiras consistem basicamente na
aplicação da chamada análise qualitativa de sistemas de equações diferenciais,
em que se busca as soluções estacionárias do problema e sua estabilidade, e a
partir disso pode-se compreender a dinâmica do sistema como um todo
\citep{murray2002}. Em problemas espacialmente explícitos, tais métodos são de
difícil implementação, mas existem métodos de homogeneização que, por vezes,
oferecem boas aproximações \citep{cobbold2014}. Embora usualmente essas
técnicas se contraponham às técnicas computacionais, várias etapas dos
cálculos analíticos, tais como encontrar soluções de equações algébricas e
autovalores, podem se beneficiar de ferramentas computacionais (por exemplo, a
biblioteca SymPy \citep{sympy}).

Já os métodos computacionais consistem, no mais das vezes, na simulação das
equações, empregando ferramentas do cálculo numérico \citep{iserles2009}. Já
temos ampla experiência no uso de bibliotecas de \emph{software} livres em
Python, como o SciPy \citep{scipy}, que provê rotinas muito robustas para
integração numérica de sistemas de equações diferenciais.


\fimdesection

\section{Objetivos específicos}
\label{sec:objetivos}

\subsubsection*{\em Modelos espacialmente estruturados de metacomunidades:
generalizando o conceito de capacidade metapopulacional}

Embora o conceito de metacomunidades seja derivado do de metapopulações,
muitos dos métodos e ideias deste nunca foram adequadamente traduzidos e
aplicados ao estudo de metacomunidades. Uma das grandes conquistas da teoria
moderna de metapopulações foi a formulação de modelos que são de grande
generalidade, mas também admitem parametrização a partir de dados específicos
de cada paisagem \citep{hanski2000}, o que tem grande valor em aplicações,
especialmente em biologia de conservação.

Baseando-nos em formulações gerais da teoria de metacomunidades
\citep{pillai2010}, visamos estender o modelo desenvolvido por
\citet{ovaskainen2001} para metapopulações, que consiste em introduzir taxas
de colonização e extinção dependentes da mancha. Isto é conveniente para
estudar paisagens específicas, em que as aproximações do modelo de Levins, de
infinitas manchas iguais, não são satisfatórias. Assim, buscamos ampliar o
quadro teórico de metacomunidades a fim de obter maior capacidade preditiva e
permitir contato mais próximo com medidas observacionais e experimentais.

\subsubsection*{\em Estados alternativos e heterogeneidade espacial promovem
coexistência de metacomunidades: o caso da predação intraguilda}

Heterogeneidade espacial é um dos principais fatores capazes de promover a
coexistência de espécies \citep{amarasekare2003}, sobretudo devido ao
mecanismo de ordenamento de espécies, em que cada espécie é superior às outras
em certas regiões. Propomos explorar essa ideia para compreender melhor
como a variação entre manchas de \emph{habitat} promove a estabilidade de
redes tróficas simples.

Em particular, pretendemos estudar o caso da predação intraguilda, um módulo
de grande importância de um ponto de vista fundamental, já que faz parte de
qualquer rede trófica, e pode provocar grande variação no comprimento e na
produtividade de cadeias tróficas, além de ter um papel importante na
manutenção da diversidade de espécies em largas escalas de espaço
\citep{mccann2011}. Num sistema com predação intraguilda, o resultado da
dinâmica (ou seja, a persistência ou não das espécies) depende da
produtividade de recursos: sob baixa produtividade, apenas a presa intraguilda
persiste, enquanto que em altas produtividades o predador intraguilda é capaz
de excluí-la, e há coexistência estável apenas em produtividades
intermediárias \citep{holt1997}. Essa conclusão é problemática na ecologia de
populações tradicional, já que a predação intraguilda é ubíqua, mas, de acordo
com a teoria, ela persistiria apenas para uma faixa de parâmetros muito
restrita.

Aqui, tomamos como objetivo desenvolver métodos para incorporar explicitamente
paisagens contendo manchas de diferentes produtividades, e explorar em que
condições se observa coexistência estável.

\subsubsection*{\em Qual o efeito de características de paisagem sobre a
dinâmica de metacomunidades?}

Finalmente, buscamos explorar como fatores físicos e comportamentais, como
tamanho de manchas e a reação de organismos à fronteira entre mancha e matriz,
pode ter consequências para a distribuição espacial dessas espécies em
diferentes escalas espaciais. A princípio, isto deve ser explorado em um
contexto mais simples, de metapopulações \citep{ovaskainen2004} (ou
``metacomunidades'' sem interações entre espécies), que deve fornecer uma base
sólida para estender essa abordagem para sistemas com interações entre as
espécies. 


%\subsubsection*{\em Predação intraguilda em manchas de produtividade heterogênea}
%
%Em primeiro lugar, escolhemos um módulo de interação -- a predação intraguilda
%-- que é de grande importância de um ponto de vista fundamental, já que faz
%parte de qualquer rede trófica, e pode provocar grande variação no comprimento
%e na produtividade de cadeias tróficas, além de ter um possível papel
%importante na manutenção da diversidade de espécies em largas escalas de
%espaço \citep{mccann2011}.
%
%A predação intraguilda é de particular interesse para este projeto porque a
%teoria tradicional prevê que o resultado da sua dinâmica depende da
%produtividade do sistema: sob baixa produtividade, apenas a presa intraguilda
%persiste, enquanto que em altas produtividades o predador intraguilda é capaz
%de excluí-la, e há coexistência estável apenas em produtividades
%intermediárias \citep{holt1997}. Essa conclusão é problemática na ecologia de
%populações tradicional, já que a predação intraguilda é ubíqua, mas, de acordo
%com a teoria, ela persistiria apenas para uma faixa de parâmetros muito
%restrita.
%
%Aqui, tomamos como objetivo desenvolver métodos para incorporar explicitamente
%paisagens contendo manchas de diferentes produtividades, e explorar em que
%condições se observa coexistência estável.
%
%\subsubsection*{\em Características de paisagem em metacomunidades}
%
%Este segundo tópico busca explorar como fatores físicos e comportamentais,
%como tamanho de manchas e a reação de organismos à fronteira entre mancha e
%matriz, pode ter consequências para a distribuição espacial dessas espécies em
%diferentes escalas espaciais. A princípio, isto deve ser explorado em um
%contexto mais simples, de metapopulações (ou ``metacomunidades'' sem
%interações entre espécies), que deve fornecer uma base sólida para estender
%essa abordagem para sistemas com interações entre as espécies. 

\fimdesection

\section{Cronograma de trabalho}

É sempre difícil apresentar um cronograma de trabalho preciso em projetos de
natureza teórica. O primeiro passo será aprofundar-se na literatura desse
tópico e o ferramental de modelagem que ele abriga. A seguir, iremos
desenvolver e refinar modelos explorando os problemas propostos na seção
\ref{sec:objetivos}. Até o fim do ano, esperamos obter resultados suficientes
para publicação.

A elaboração de cada modelo é um processo criativo, que envolve idas e vindas,
já que em cada passo é necessário analisar as hipóteses e suas consequências,
usando os métodos analíticos e computacionais descritos, alterando formas de
descrição, hipóteses e análises, até alcançar modelos que sejam informativos,
porém suficientemente simples para que sua interpretação seja clara. 

Naturalmente, o candidato deve dar continuidade a colaborações nacionais e
internacionais que têm se mostrado produtivas.

\begin{table}[htb] 
\small
\noindent
\begin{tabular}{p{6cm}|p{0.3cm}p{0.3cm}p{0.3cm}p{0.3cm}p{0.3cm}p{0.3cm}p{0.3cm}p{0.3cm}p{0.3cm}p{0.3cm}p{0.3cm}p{0.3cm}}
    {\normalsize Atividades\textbackslash\hspace{0.1cm} Bimestre} & 1 & 2 & 3 & 4 & 5 & 6 & 7 & 8
                                                                & 9 & 10 & 11 & 12 \\ \hline 
                                              revisão da literatura& \cellcolor[gray]{0.8} & \cellcolor[gray]{0.8} &  &  &  &  &  &  &  &  &  &   \\
                                    modelo predação intraguilda &
                                          \cellcolor[gray]{0.8} &
                                          \cellcolor[gray]{0.8} &
                                          \cellcolor[gray]{0.8} &
                                          \cellcolor[gray]{0.8} &
                                          \cellcolor[gray]{0.8} & & & & & & & \\
                                    \mbox{modelos espacialmente estruturados} & & & &
                                          \cellcolor[gray]{0.8} &
                                          \cellcolor[gray]{0.8} &
                                          \cellcolor[gray]{0.8} &
                                          \cellcolor[gray]{0.8} &
                                          \cellcolor[gray]{0.8} & & & & \\
                                    \mbox{modelos de paisagens para metacom.} & & & & & & &
                                          \cellcolor[gray]{0.8} &
                                          \cellcolor[gray]{0.8} &
                                          \cellcolor[gray]{0.8} &
                                          \cellcolor[gray]{0.8} &
                                          \cellcolor[gray]{0.8} & \\
                                     \mbox{exploração conexão entre modelos} & & &
                                          \cellcolor[gray]{0.8} &
                                          \cellcolor[gray]{0.8} &
                                          \cellcolor[gray]{0.8} &
                                          \cellcolor[gray]{0.8} &
                                          \cellcolor[gray]{0.8} &
                                          \cellcolor[gray]{0.8} &
                                          \cellcolor[gray]{0.8} &
                                          \cellcolor[gray]{0.8} &
                                          \cellcolor[gray]{0.8} & \\ 
                                    escrita dos resultados & & & & & 
                                          \cellcolor[gray]{0.8} &
                                          \cellcolor[gray]{0.8} & & & & &
                                          \cellcolor[gray]{0.8} &
    \cellcolor[gray]{0.8} \\
\end{tabular}
\end{table}

\fimdesection

\section{Considerações finais} % justificativa?

O candidato desenvolve desde sua graduação pesquisa em sistemas dinâmicos em
biologia, buscando ativamente colaboração com pesquisadores da ecologia. Entre
os resultados há artigos em periódicos de alto impacto da área
\citep{coutinho2012, laporta2013, assaneo2013, fonseca2013, amarasekare2013,
moretti2013, azevedo2014, amarasekare2014} e participação como professor
assistente nas \emph{Southern Summer Schools on Mathematical Biology}
\citep{SSSMB2012, SSSMB2013, SSSMB2014}, que promovem a interação entre
estudantes das ciências exatas e biológicas. O supervisor da proposta é hoje
um dos colaboradores principais do candidato e busca, igualmente, impulsionar
o diálogo com pesquisadores das ciências exatas para avanço da teoria. A
presente proposta tem o objetivo de estreitar esta colaboração e consolidar a
formação do candidato na área de ecologia teórica, com benefícios de parte a
parte.

Este projeto fará uso de diversas técnicas provenientes da teoria de sistemas
dinâmicos para avançar a compreensão de metacomunidades. Assim, a formação
interdisciplinar do candidato é uma grande oportunidade, já que a familiaridade com
modelos em ecologia permite navegar nessa área, que apresenta uma vasta gama de
abordagens teóricas e de argumentos e hipóteses cuja tradução matemática pode
ser sutil.


\bibliographystyle{amnatnat}
\bibliography{referencias,/home/renato/Dropbox/posdoc/sumula/minhas_citacoes}

\vskip 3.0cm
{\hfill S\~ao Paulo, \today}

{\hfill  \it Paulo Inácio de Knegt López de Prado}

{\hfill  \it Renato Mendes Coutinho}

\end{document}

