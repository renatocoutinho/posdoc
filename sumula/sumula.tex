\documentclass[a4paper ,11pt]{article}
\RequirePackage{moderncvcollection}
\usepackage[brazil]{babel}
\usepackage[utf8]{inputenc}
\usepackage{hyperref}
\usepackage{geometry}
\usepackage{xcolor}

% Uncomment the following lines to use the Palatino font.  Remove the
% [osf] bit if you don't like the old style figures.
%
\usepackage[T1]{fontenc}
\usepackage[osf]{mathpazo}

\def\name{Renato Mendes Coutinho}

% The following metadata will show up in the PDF properties
\hypersetup{
    colorlinks = true,
    urlcolor = black,
    pdfauthor = {\name},
    pdfkeywords = {ecologia teórica},
    pdftitle = {\name: Curriculum Vitae},
    pdfsubject = {Curriculum Vitae},
    pdfpagemode = UseNone
}

\geometry{textheight=8.5in, textwidth=6in}

% Customize page headers
\pagestyle{empty}
%\markright{\name}
%\thispagestyle{empty}

% Customize section headings
\usepackage{sectsty}
\subsectionfont{\rmfamily\mdseries\itshape\large}

% Don't indent paragraphs.
\setlength\parindent{0em}

% Make lists without bullets
\renewenvironment{enumerate}{
  \begin{list}{}{
    \setlength{\leftmargin}{2em}
  }
}{
  \end{list}
}

\usepackage{pifont}
\renewcommand{\labelitemi}{\ding{226}}

%%% excerpts from moderncv
\RequirePackage{calc}
\RequirePackage{xparse}
\newlength{\separatorcolumnwidth}
\setlength{\separatorcolumnwidth}{0.025\textwidth}
\newlength{\hintscolumnwidth}
\setlength{\hintscolumnwidth}{0.175\textwidth}
\newlength{\maincolumnwidth}
\setlength{\maincolumnwidth}{\textwidth-\separatorcolumnwidth-\hintscolumnwidth}%
\newcommand*{\hintfont}{}
\definecolor{color0}{rgb}{0,0,0}% main default color, normally left to black
\newcommand*{\hintstyle}[1]{{\hintfont\textcolor{color0}{#1}}}

\newcommand*{\cvitem}[3][.25em]{%
  \begin{tabular}{@{}p{\hintscolumnwidth}@{\hspace{\separatorcolumnwidth}}p{\maincolumnwidth}@{}}%
    \raggedleft\hintstyle{#2} &{#3}%
  \end{tabular}%
  \par\addvspace{#1}}

\newcommand*{\cventry}[7][.25em]{%
  \cvitem[#1]{#2}{%
    {\bfseries#3}%
   \ifthenelse{\equal{#4}{}}{}{, {\slshape#4}}%
    \ifthenelse{\equal{#5}{}}{}{ (#5)}%
    \ifthenelse{\equal{#6}{}}{}{, #6}%
    .\strut%
    \ifx&#7&%
      \else{\newline{}\begin{minipage}[t]{\linewidth}\small#7\end{minipage}}\fi}}
% usage: \cventry[spacing]{years}{degree/job title}{institution/employer}{localization}{optionnal: grade/...}{optional: comment/job description}

%%%%

\begin{document}

\centerline{\Large Súmula Curricular}
\vspace{2em}
\centerline{\huge\bf \name}

\vspace{.5cm}

%\begin{minipage}[t]{0.5\textwidth}
%    \href{http://www.ift.unesp.br/}{IFT/Unesp} - São Paulo, SP \\
%    Data de Nascimento: 15 de Outubro de 1985 \\
%    Nacionalidade: Brasileira
%\end{minipage}
%\begin{minipage}[t]{0.5\textwidth}
%    Telefone: 11 3624 8308\\
%    Celular: 11 9694 7061 \\
%    Email: \href{mailto:renatomc@ift.unesp.br}{\tt renatomc@ift.unesp.br} \\
%\end{minipage}

\section{Formação Acadêmica}

\cventry{2004--2008}{Graduação em Ciências Moleculares}{Universidade de São Paulo}{São Paulo}{}{}
\cventry{2008--2010}{Mestrado em Física}{Instituto de Física Teórica -- Unesp}{São
Paulo}{Dissertação: ``Equações diferenciais com retardo em biologia de
populações''}{Orientador: Roberto Kraenkel. Bolsista do CNPq.}
\cventry{2010--}{Doutorado em Física}{Instituto de Física Teórica -- Unesp}{São
Paulo}{Tese: ``Dinâmica de populações estruturadas''}{Orientador: Roberto
Kraenkel. Bolsista da FAPESP}
\cventry{2012--2013}{Estágio Sanduíche}{Institute of Biochemistry and Biology
-- University of Potsdam}{Potsdam,
Alemanha}{Projeto: ``Trait dynamics in interacting populations''}{Supervisora: Ursula Gaedke.
Bolsista da FAPESP.}

%\begin{enumerate}
%    \item Bacharelado em Ciências Moleculares pela
%        Universidade de São Paulo de 2004 a 2008.
%        \begin{enumerate}
%            \item Bolsista de iniciação científica do CNPq de 09/2004 a
%                06/2006 sob orientação da professora Alinka Lépine.
%            \item Aluno de iniciação científica de 08/2006 a
%                06/2008 sob orientação do professor Marcus Aguiar.
%        \end{enumerate}
%    \item Atualmente concluindo o Mestrado iniciado em 08/2008 no Instituto de
%        Física Teórica (IFT) da Unesp sob orientação do professor Roberto André
%        Kraenkel, com bolsa CNPq desde 07/2009.
%\end{enumerate}

\section{Histórico profissional}
Nada a declarar

\section{Produção científica relevante}

\renewcommand{\refname}{Artigos}

\nocite{*}
\bibliographystyle{abbrv}
\bibliography{minhas_citacoes}

\section{Financiamentos à pesquisa vigentes}

Bolsa de doutorado regular pela FAPESP (vigência de 09/2010--08/2012 e 09/2013--02/2015).

\section{Orientações em andamento}
Nada a declarar

\section{Indicadores quantitativos}
\begin{enumerate}
    \item Livros publicados: \textbf{0}
    \item Publicações em periódicos: \textbf{8}
    \item Capítulos de livros: \textbf{0}
    \item Teses de mestrado orientadas: \textbf{0}
    \item Teses de doutorado orientadas: \textbf{0}
    \item Citações (Google Scholar): \textbf{15}
\end{enumerate}

\section{\emph{My Citations} Google Scholar}
\url{http://scholar.google.com.br/citations?user=RkgQvSYAAAAJ&hl=pt-BR}

\section{Outras informações}
\begin{itemize}
    \item Monitor das escolas de verão \emph{Métodos matemáticos em biologia de
populações} de 2009, 2010 e 2011 e das \emph{Southern Summer Schools on Mathematical
Biology} de 2012 e 2013, e professor assistente na edição de 2014.
    \item Monitor do curso de Ecologia de Populações de 2014 oferecida pela Pós-Graduação do Departamento de
Ecologia do IB-USP
    \item Colaboração desde 2012 com a prof. Priyanga Amarasekare, da UCLA
\end{itemize}

\end{document}

